% simple structure that shows joined rings and coloured atoms, as well as use of
% the \chemabove{}{} macro
\chemstruct{
    *6(
        -
        =
        -
        (-\textcolor{N}{N}*6(
            -
            -
            -\textcolor{N}{N}(
                -[1]
                -[-1]
                -[1]
                -[-1]
                -[1]\textcolor{O}{O}
                -[-1]*6(
                    -
                    =
                    -
                    (*6(
                        -
                        -
                        -
                        (=\textcolor{O}{O})
                        -\textcolor{N}{\chemabove{N}{H}}
                        -
                        -[,,,,draw=none]
                    ))
                    =
                    -
                    =
                )
            )
            -
            -
            -))
        =
        (-\textcolor{Cl}{Cl})
        -
        (-\textcolor{Cl}{Cl})
        =
    )
}

% structure showing the use of a single hook (?) to join distant atoms, in this
% case to make a ring structure not using the * ring options
% also shows how to alter colour, style and type of bonds
% shows the use of an invisible bond to place a stereocentre indicator
\chemstruct{
    ?-[1.5,,,,t]
    -[-1,,,,t]
    -[0,,,,t](-[3,,,,t]\textcolor{Ot}{O}|\textcolor{Ot}{H})(
        -[-1,,,,t](
            -[-3]*6(
                -
                =
                -(
                    -[,,,,t]\textcolor{Ot}{O}
                    -[-1,,,,t]\textcolor{t}{C}|\textcolor{t}{H_3}
                )
                =
                -
                =
            )
        )
        (-[3,0.33,,,draw=none]\textcolor{t}{*})
        -[1]
        -[-1]\textcolor{N}{N}
        (-[-3,,,,t]\textcolor{t}{C}|\textcolor{t}{H_3})
        -[1,,,,t]\textcolor{t}{C}|\textcolor{t}{H_3}
    )
    <[-4.5,,,,t]
    -[5,,,,t,line width = 0.8ex, shorten <= -0.33ex, shorten >= -0.33ex]
    ?[,{<},{t}]
}

% shows the use of wedge and dash bonds as well as the stereochemical marker
\chemstruct{
    \textcolor{N}{N}~[1]
    -[1]*6(
        -
        =(*5(
            -
            -\textcolor{O}{O}
            -
            (-[-3.75,0.5,,,draw=none]\stereomarker{S})
            (<:[4]*6(
                -
                =
                -
                (-\textcolor{F}{F})
                =
                -
                =
            ))
            (
                <[1]
                -[-1]
                -[1]
                -[-1]\textcolor{N}{N}
                (-[-3]CH_3)
                -[1]CH_3
            )
            -
            -[,,,,draw=none]
        ))
        -
        =
        -
        =
    )
}

% shows the basics of a scheme, including electron movement, arrows and reaction
% conditions, as well as aligning of molecules in a scheme
\chemscheme{
    \schemestart
        \chemfig{
            *6(
                (-[-3]H)
                -[@{b1},,,,draw=none]
                =[@{d1}]
                -[@{s1}]
                (~[@{t1}3]\charge{90=\:}{\textcolor{N}{N}})
                -[,,,,draw=none]@{b2}
                -[,,,,draw=none]
                (-[3]H)
                ~[@{t2}]
            )
        }
        \arrow(c2.base east--c2.base west){<=>[\cond{Δ}][][8.4pt]}
        \chemleft[
            \chemfig{
                *6(
                    -
                    -
                    =
                    =\charge{90=\:}{\textcolor{N}{N}}
                    -
                    =
                )
            }
        \chemright]
        \arrow(c2.base east--c2.base west){->[\cond{±\ce{H+}}][][8.4pt]}
        \chemfig{
            *6(
                -
                =
                -
                =\charge{90=\:}{\textcolor{N}{N}}
                -
                =
            )
        }
    \schemestop
    \chemmove[
        arrows = {-Triangle[scale = 0.67]},
        shorten <= 0.77ex,
        shorten >= 0.1ex,
        line width = 0.1em,
        r
    ]{
        \draw (d1) .. controls +(100:1em) and +(80:1em) .. (b1);
        \draw (t1) .. controls +(180:1em) and +(-180:1em) .. (s1);
        \draw (t2) .. controls +(-20:1em) and +(-90:1em) .. (b2);
    }
}

% shows the use of the \chemmove{} macro to draw on top of the structure,
% as well as some tricks using invisible bonds to generate correctly placed
% nodes for ligands with hapticity greater than unity
% shows how to draw lone pairs on an atom
\chemscheme{
    \chemfig{
        \textcolor{Fe}{Fe}
        (-[3,1.25,,,]@{top}
        -[3,0.4,,,draw=none](
            [1]-[-0.5,,,,]
            (-[0](=[2]\textcolor{O}{O})-[-2]C|H_3)
            <[-4,0.67]?[1]
        )(
            -[-5.5]
            <[-2,0.67]
            ?[1,,{line width=0.8ex},{shorten <= -0.33ex},{shorten >= -0.33ex}]
        ))
        (-[-3,1.15,,,]@{bottom}
        -[-3,0.5,,,draw=none](
            >[0.5,,,,]
            -[4,0.67]?[2]
        )(
            >[5.5,,,,]
            (-[6](=[4]\textcolor{O}{O})-[-4,,,2]H_3|C)
            -[2,0.67]
            ?[2,,]
        ))
    }
    \chemmove[line width = 0.1em]{
        \draw (top) ellipse (1em and 0.4em);
        \draw (bottom) ellipse (1em and 0.4em);
    }
}

% an example showing the more complex use of mutiple hooks to generate a multi-
% cyclic, non-planar structure
\chemstruct{
    ?[1]-[2](
        -[3]
        -[0.5]?[2]
        -[3]\charge{90=\:}{\textcolor{N}{N}}|\textcolor{N}{H_2}
    )
        (-[-5.5]H_3|C)
    -[-0.5]
    -[0.5]
        (-[3]?[2])
        (-[-0.5]C|H_3)
    -[-4]
    -[5.5]
    ?[1]
        (-[3]?[2])
}

% structure showing an octahedral metal complex with an overall charge
% utilises invisible bonds for alignment and hooks for creating bonds
% shows how to construct a dative bond
\chemscheme{
    \schemestart
        \chemleft[
        \chemfig{
            @{Ru}\textcolor{Ru}{Ru}-[-1,2,,,draw=none]@{N1}\textcolor{N}{N}**6(
                -
                -
                -
                -
                -(-[,,,,draw=none]**6(
                    -
                    -
                    -
                    -(-[,,,,]**6(
                        -
                        -
                        -
                        -(-[,,,,draw=none]**6(
                            -
                            -
                            -
                            -(-[,,,,]**6(
                                -
                                -
                                -
                                -(-[,,,,draw=none]**6(
                                    -
                                    -
                                    -
                                    -(?[rings])
                                    -@{N2}\textcolor{N}{N}
                                    -
                                ))
                                -@{N3}\textcolor{N}{N}
                                -
                            ))
                            -@{N4}\textcolor{N}{N}
                            -
                        ))
                        -@{N5}\textcolor{N}{N}
                        -
                    ))
                    -@{N6}\textcolor{N}{N}
                    -
                ))
                -
            )
        }
        \chemright]
        \arrow{0}[,0,]
        \chemleft[
        \chemfig{
            \ce{\textcolor{Cl}{Cl}-}
        }
        \chemright]
    \schemestop
    \chemmove[
        arrows = {-Triangle[scale = 0.67]},
        shorten <= 0.33ex,
        shorten >= 0.33ex,
        line width = 0.1em
    ]{
        \draw (N1) -- (Ru);
        \draw (N2) -- (Ru);
        \draw (N3) -- (Ru);
        \draw (N4) -- (Ru);
        \draw (N5) -- (Ru);
        \draw (N6) -- (Ru);
    }
   \chemmove{
        \node[xshift = 1ex] at (c1.north east) {2+};
        \node[xshift = 0.5ex] at (c2.south east) {2};
   }
}
