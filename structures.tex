% -  single
% =  double
% ~  triple
% <  cram
% <: dash
% bond arguments [dir in n*30deg,length,start atom,end atom,tikz code]
% *n() for n size ring, **n() for n size ring with circle

\chemstruct{
    *6(
        -
        =
        -
        (-\textcolor{N}{N}*6(
            -
            -
            -\textcolor{N}{N}(
                -[1]
                -[-1]
                -[1]
                -[-1]
                -[1]\textcolor{O}{O}
                -[-1]*6(
                    -
                    =
                    -
                    (*6(
                        -
                        -
                        -
                        (=\textcolor{O}{O})
                        -\textcolor{N}{\chemabove{N}{H}}
                        -
                        -[,,,,draw=none]
                    ))
                    =
                    -
                    =
                )
            )
            -
            -
            -))
        =
        (-\textcolor{Cl}{Cl})
        -
        (-\textcolor{Cl}{Cl})
        =
    )
}

\chemstruct{
    ?-[1,,,,t]
    -[-1,,,,t]
    -[0,,,,t](-[3,,,,t]\textcolor{Ot}{O}|\textcolor{Ot}{H})(
        -[-1,,,,t](
            -[-3]*6(
                -
                =
                -(
                    -[,,,,t]\textcolor{Ot}{O}
                    -[-1,,,,t]\textcolor{t}{C}|\textcolor{t}{H_3}
                )
                =
                -
                =
            )
        )
        (-[3,0.33,,,draw=none]\textcolor{t}{*})
        -[1]
        -[-1]\textcolor{N}{N}
        (-[-3,,,,t]\textcolor{t}{C}|\textcolor{t}{H_3})
        -[1,,,,t]\textcolor{t}{C}|\textcolor{t}{H_3}
    )
    <[-5,,,,t]
    -[5,,,,t,line width = 0.8ex, shorten <= -0.33ex, shorten >= -0.33ex]
    ?[,{<},{t}]
}

\chemstruct{
    *6(
        -[,,,,]
        =[,,,,]
        -[,,,,](
            -[1,,,,]
            -[-1,,,,]
            (-[-5,0.33,,,draw=none]\textcolor{t}{*})
            (-[-3,,,,t]\textcolor{t}{C}|\textcolor{t}{H_3})
            -[1,,,,]\textcolor{N}{\chemabove{N}{H}}
            -[-1,,,,t]\textcolor{t}{C}|\textcolor{t}{H_3}
        )
        =[,,,,]
        -[,,,,]
        (*5(
            -[,,,,t]\textcolor{Ot}{O}
            -[,,,,t]
            -[,,,,t]\textcolor{Ot}{O}
            -[,,,,t]
            -[,,,,draw=none]
        ))
        =[,,,,]
    )
}

\chemstruct{
    \textcolor{N}{N}~[1]
    -[1]*6(
        -
        =(*5(
            -
            -\textcolor{O}{O}
            -
            (-[-3.75,0.5,,,draw=none]\stereomarker{S})
            (<:[4]*6(
                -
                =
                -
                (-\textcolor{F}{F})
                =
                -
                =
            ))
            (
                <[1]
                -[-1]
                -[1]
                -[-1]\textcolor{N}{N}
                (-[-3]CH_3)
                -[1]CH_3
            )
            -
            -[,,,,draw=none]
        ))
        -
        =
        -
        =
    )
}

\chemscheme{
    \schemestart
        \chemfig{
            *6(
                (-[-3]H)
                -[@{b1},,,,draw=none]
                =[@{d1}]
                -[@{s1}]
                (~[@{t1}3]\textcolor{N}{N})
                -[,,,,draw=none]@{b2}
                -[,,,,draw=none]
                (-[3]H)
                ~[@{t2}]
            )
        }
        \arrow(c2.base east--c2.base west){<=>[\cond{Δ}][][8.4pt]}
        \chemleft[
            \chemfig{
                *6(
                    -
                    -
                    =
                    =\textcolor{N}{N}
                    -
                    =
                )
            }
        \chemright]
        \arrow(c2.base east--c2.base west){->[\cond{±\ce{H+}}][][8.4pt]}
        \chemfig{
            *6(
                -
                =
                -
                =\textcolor{N}{N}
                -
                =
            )
        }
    \schemestop

    \chemmove[shorten <= 0.77ex, shorten >= 0.1ex, line width = 0.067em, r]{
        \draw (d1) .. controls +(100:1em) and +(80:1em) .. (b1);
        \draw (t1) .. controls +(180:1em) and +(-180:1em) .. (s1);
        \draw (t2) .. controls +(-20:1em) and +(-90:1em) .. (b2);
    }
}

\chemscheme{
    \chemfig{
        \textcolor{Fe}{Fe}
            (-[3,1.25,,,]@{top}
            -[3,0.4,,,draw=none]
                (
                    [1]-[-0.5,,,,]
                    (-[0](=[2]\textcolor{O}{O})-[-2]C|H_3)
                    <[-4,0.67]?[1]
                )
                (
                    -[-5.5]
                    <[-2,0.67]
                    ?[1,,{line width=0.8ex},{shorten <= -0.33ex},{shorten >= -0.33ex}]
                )
            )
            (-[-3,1.15,,,]@{bottom}
            -[-3,0.5,,,draw=none]
                (
                    >[0.5,,,,]
                    -[4,0.67]?[2]
                )
                (
                    >[5.5,,,,]
                    (-[6](=[4]\textcolor{O}{O})-[-4,,,2]H_3|C)
                    -[2,0.67]
                    ?[2,,]
                )
            )
    }
    
    \chemmove[line width = 0.1em]{
        \draw (top) ellipse (1em and 0.4em);
        \draw (bottom) ellipse (1em and 0.4em);
    }
}

\chemstruct{
    ?[1]-[2](
        -[3]
        -[0.5]?[2]
        -[3]\charge{90=\:}{\textcolor{N}{N}}|\textcolor{N}{H_2}
    )
        (-[-5.5]H_3|C)
    -[-0.5]
    -[0.5]
        (-[3]?[2])
        (-[-0.5]C|H_3)
    -[-4]
    -[5.5]
    ?[1]
        (-[3]?[2])
}
